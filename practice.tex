% practice.tex - Our first LaTeX example!

\documentclass{article}

% preamble - start
\usepackage{textcomp}
\usepackage{hyperref} % makes ref clickable
% preamble - end

\begin{document}

\title{This is my title}
\author{Edwin Torres}

\maketitle

\tableofcontents  % Typeset twice, because this takes the previous compile cycle

\newpage

This is how you \underline{underline}.
This is how you \emph{emphasize}. % i.e. italicize


I like BASIC\@. What about you? % ends sentence after C
I like BASIC. What about you? % does not end sentence after C

% avoiding ligatures (notice how ff run together in first word)
\Large shelfful shelf\mbox{}ful

% this is different than using three dots (...)
Here is an ellipsis \ldots

% textcomp package lets's us do this...
30 \textcelsius{} is
86 \textdegree{}F.

% important: specifying single and double-quotes:
``Please press the `x' key.''   ABC 

% apostrophe
It\textsc{\char13}s a nice day!

It\textsc{\char13}s $-30\,^{\circ}\mathrm{C}$.
I will soon start to
super-conduct.



 read\slash write

\section{Example 1}
% Example 1
\ldots when Einstein introduced his formula
\begin{equation}
 e = m \cdot c^2 \; ,
\end{equation}
which is at the same time the most widely known
and the least well understood physical formula.

A reference to this subsection \label{sec:this} 
looks like: ``see Section~\ref{sec:this}. on
Page~\pageref{sec:this}.''


\section{Example 2}
% Example 2
\ldots from which follows Kirchhoffs current law:
\begin{equation}
 \sum_{k=1}^{n} I_k = 0 \; .
\end{equation}


Kirchhoffs�� voltage law can be derived \ldots

\section{Example 3}
% Example 3
\ldots which has several advantages.

\begin{equation}
 I_D = I_F - I_R
\end{equation}
is the core of a very different transistor model. \ldots

This is an example of a footnote\footnote{This word has medieval origins.} .
I like to catch fluke\footnote{The fluke is also known as the summer flounder.} at the Jersey Shore .






\end{document}
